\TheCommand{buildin}%
\Command{buildin name: name [, args: value...] [, args...]}

\TypeTable{buildin \\}{%
% properties
theCommandName \\
theCommand \\
%\midrule
% operators
}

Execute the specified build-in command and pass the optional arguments
to the command. This is useful if there is a function with the same name
as the build-in declared.
The returned command delegates all unknown methods and properties to the
executed build-in command.

\begin{asparadesc}
%
\item[\code{name: name}]  \hfill \\
the name of the build-in command.
%
\item[\code{theCommandName}]  \hfill \\
the name of the command that was created.
%
\item[\code{theCommand}]  \hfill \\
returns the executed build-in command.
%
\end{asparadesc}

\begin{lstlisting}[style=Groovybash, label={lst:example_buildin}, title={
Call build-in commands.}]
buildin name: "echo", "Test"
buildin name: "echo", newline: false, "Test"
\end{lstlisting}

\begin{lstlisting}[style=Groovybash, label={lst:example_buildin_override}, title={
Override build-in command.}]
def cd() {
    buildin name: "cd"
}

cd
\end{lstlisting}

\begin{lstlisting}[style=Groovybash, label={lst:example_buildin_delegate}, title={
Delegate the properties to the executed build-in command.}]
def files = buildin name: "listFiles" theFiles
\end{lstlisting}

