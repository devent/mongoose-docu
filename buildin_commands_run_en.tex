\TheCommand{run}%
\Command{run [handler,] [, destroyer] [, watchdog] [, timeout] cmd [, env] \\~ [, directory]}

\TypeTable{run \\}{%
% properties
handler \\
destroyer \\
watchdog \\
timeout \\
env \\
directory \\
%\midrule
% operators
}

Executes the specified external command in a separate process with the
specified environment and working directory. It can not execute build-in commands.

\begin{asparaitem}
%
\item[\code{handler: ExecuteResultHandler}] \hfill \\
a callback that is called when the process completes. 
The methods of the \Type{ExecuteResultHandler}\cite{executeresulthandler13} 
are called when the process completes or when the process fails.

Can be set to an instance of the \Type{ExecuteResultHandler} or of a class 
type; the class type is instantiated with the default
constructor and set as the handler.

An handler must be set if the command is executed in the background by the \Operator*{background()}
operator, otherwise it is possible that the script will finish before the command finishes.
%
\item[\code{destroyer: ProcessDestroyer}] \hfill \\
\Type{ProcessDestroyer}\cite{processdestroyer13}  destroys the process under certain conditions.
Defaults to the \Type{ShutdownHookProcessDestroyer}\cite{shutdownhookprocessdestroyer13}
that will destroy the process after the script terminates unexpectelly.
%
\item[\code{watchdog}] \hfill \\
destroys the process after a specified timeout. Defaults to 
no watchdog which means the process will never timeout.
%
\item[\code{timeout: (+)number|duration}] \hfill \\
sets the timeout for the process. Defaults to an infinite 
timeout \code{ExecuteWatchdog} \code{\#INFINITE\_TIMEOUT}. Can be set to a positive
number or a \Type{Duration}\cite{duration13}.
%
\item[\code{env}] \hfill \\
the system-dependent mapping from variables
to values. The initial value is a copy of the environment of the current
script. Can be a string containing \code{name=value} tupels
separated by a comma, or a map containing \code{key=value} entries.
%
\item[\code{dir}] \hfill \\
the working directory of the process. The default value
is the current working directory set by the \Command*{cd} command
and saved in the \Variable*{PWD} variable.
%
\end{asparaitem}

\begin{lstlisting}[style=Groovybash, label={lst:example_run1}, title={
Examples to execute external commands.}]
run "ls -al"
run "javap -private", "path=/home/user/java6/bin"
run "mvn package", [JAVA_HOME: "/home/user/java6"]
\end{lstlisting}

\begin{lstlisting}[style=Groovybash, label={lst:example_run2}, title={
Deleyed execution of external commands.}]
def command = run.args "ls -al"
command.directory = "/tmp"
command.timeout = 500
command()
\end{lstlisting}

