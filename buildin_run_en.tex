%%
\label{par:run}
\parparameter{\parcommand{run} [redirectErrorStream: true|false,] \partype{cmd} [, \partype{envp}] [, \partype{dir}]}

Executes the specified command in a separate process with the
specified environment and working directory.

\subparagraph{Flags:}

\begin{asparaitem}
%
\item[\code{redirectErrorStream: true|false}] Specify \code{true} that
the standard error is merged with the standard output. Defaults to \code{false}
meaning that the standard output and error output of the command are sent to
two separate streams.
%
\end{asparaitem}

\subparagraph{Arguments:}

\begin{asparaitem}
%
\item[\partype{cmd}] the command and the arguments for the command, if any.
%
\item[\partype{envp}] which is a system-dependent mapping from variables
to values. The initial value is a copy of the environment of the current
process. Can be a \code{String} containing \emph{name=value} tupels
separated by a comma, or a \code{Map} containing \emph{key=value} entries.
%
\item[\partype{dir}] the working directory of the process. The default value
is the current working directory set by \parcommand{cd}, see \refpar{par:cd}.
%
\end{asparaitem}

\subparagraph{Example:}

\begin{lstlisting}[style=Groovybash, label={lst:example_exit}]
run "ls -al"
run "javap -private", "path=/home/user/java6/bin"
run "mvn package", [JAVA_HOME: "/home/user/java6"]
run out: "out.txt", redirectErrorStream: true, "ls -al"
\end{lstlisting}

