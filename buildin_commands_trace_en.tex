\TheCommand{trace}%
\Command{trace message [, args]}

\TypeTable{trace \\}{%
% properties
theContext \\
isEnabled \\
%\midrule
% operators
}

Log a message at the trace level of the logger.
The logging facade Slf4j\cite{slf4j13} is used and the used logger is
the global script logger.

\begin{asparadesc}
%
\item[\code{message}] \hfill \\
The message.
%
\item[\code{args}] \hfill \\
The arguments for the message.
%
\item[\code{theContext}] \hfill \\
The name of the logger that is used. Should return the class name of the current
script.
%
\item[\code{isEnabled}] \hfill \\
Tests if the trace level is enabled. Is \code{true} if the trace level is enabled;
\code{false} if not.
%
\end{asparadesc}

\begin{lstlisting}[style=Groovybash, label={lst:example_trace1}, title={%
Outputs a trace logging message with arguments.}]
trace "The trace message."
trace "The variable {} should be {}.", variable, expected
trace "The variable $variable should be $expected."
\end{lstlisting}

\begin{lstlisting}[style=Groovybash, label={lst:example_trace2}, title={%
Prints the name of the current logger.}]
echo trace.theContext
\end{lstlisting}

\begin{lstlisting}[style=Groovybash, label={lst:example_trace3}, title={%
Tests if the trace level is enabled.}]
if (trace.isEnabled) {
    trace "The trace message."
}
\end{lstlisting}

