\TheCommand{exec}%
\Command{exec [handler: handler] [destroyer: destroyer] [watchdog: \\
watchdog] [timeout: duration] [successExitValue: number]\\~
[successExitValues: list] command [env] [path]}

\TypeTable{exec \\}{%
% properties
handler \\
destroyer \\
watchdog \\
timeout \\
env \\
directory \\
successExitValue \\
successExitValues \\
theExitValue \\
%\midrule
% operators
}

Executes the specified external \Arg{command} in a separate process with the
specified environment and working directory. The \Arg{command} contains the
name of the command to be looked up by the system or the full path of the command.
Any arguments for the command should be specified in \Arg{command}.

The \Arg{handler} is called when the process completes or when the process 
fails. The success return value
or a list of possible success return values of the process can be set by 
the \Arg{successExitValue} or by the \Arg{successExitValues} argument.
An handler must be set if the command is executed in the background by the \Operator*{background()}
operator, otherwise it is possible that the script will finish before the command itself finishes,
resulting in killing the command forcefully.

The \Arg{destroyer} can terminate the process under certain conditions.
The default destroyer will terminate the process if the JVM terminates under
any circumdances, see \code{Runtime}\code{\#addShutdownHook(Thread)}\cite{addshutdownhook13}.

An timeout can be set by the \Arg{timeout} argument. For special conditions
that can trigger a timeout the \Arg{watchdog} can be set. The default settings
is no timeout, meaning the build-in will wait for the process to terminate.

The environment as the system-dependent mapping from variables
to values can be set with the \Arg{env} argument. The initial value is a copy 
of the environment of the current script \Variable*{ENV}. Can be set to a map 
containing \code{key=value} entries.
The working directory of the process can be set by \Arg{path}. The default value
is the current working directory set by the \Command*{cd} command
and saved in the \Variable*{PWD} variable.

\begin{asparaitem}
%
\item[\code{handler: handler}] \hfill \\
the \Type{ExecuteResultHandler}\cite{executeresulthandler13} or of a class 
type; the class type is instantiated with the default
constructor and set as the handler. Defaults to the 
\Type{DefaultExecuteResultHandler}\cite{defaultexecuteresulthandler13}.
%
\item[\code{destroyer: destroyer}] \hfill \\
the \Type{ProcessDestroyer}\cite{processdestroyer13} or of a class 
type; the class type is instantiated with the default
constructor and set as the handler.
Defaults to the \Type{ShutdownHookProcessDestroyer}\cite{shutdownhookprocessdestroyer13}
%
\item[\code{watchdog: watchdog}] \hfill \\
the \Type{ExecuteWatchdog}\cite{executewatchdog13} or of a class 
type; the class type is instantiated with the default
constructor and set as the handler.
Defaults to no watchdog.
%
\item[\code{timeout: duration}] \hfill \\
the timeout in milliseconds or of \Type{Duration}\cite{duration13}.
Defaults to \code{ExecuteWatchdog}\code{\#INFINITE\_TIMEOUT}.
%
\item[\code{command}] \hfill \\
the command to execute, with additional arguments.
%
\item[\code{env}] \hfill \\
map containing \code{key=value} entries for the environment.
%
\item[\code{path}] \hfill \\
the path of the working directory for the process.
%
\end{asparaitem}

\begin{lstlisting}[style=Groovybash, label={lst:example_exec1}, title={
Examples to execute external commands.}]
exec "ls -al"
exec "javap -private", "/home/user/java6/bin"
exec "mvn package", [JAVA_HOME: "/home/user/java6"]
\end{lstlisting}

\begin{lstlisting}[style=Groovybash, label={lst:example_exec2}, title={
Deleyed execution of external commands.}]
def command = exec.args "ls -al"
command.directory = "/tmp"
command.timeout = 500
command()
\end{lstlisting}

