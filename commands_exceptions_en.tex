\subsection{Exceptions}

Instead of return values all commands can throw an exception if they encounter
any errors. In the Java language there are two kind of exceptions:
unchecked exceptions and checked exceptions.

All unchecked exceptions will be treaded as user errors in the usage of the
commands, i.e. bugs in the code: wrong arguments, \code{null} values for mandatory arguments,
wrong type of arguments, syntax errors, and so on. The unchecked exceptions
can be mitigated by fixing the code, using default values for user specified 
command line parameter or by test if the argument is valid for the command.

Checked exceptions are treaded as expected errors, that is, errors that are ancipiated
and where the user can decide what as an alternative can be done. File not found
errors can be mitigated by asking the user for an alternative file; I/O errors
can be mitigated by informing the user so the user can check the intergrity
of the hard disk.

The Groovy language does not have the concept of checked/unchecked exceptions,
all exceptions are unchecked making catching exceptions optional. Any uncatched
exceptions will terminate the current script and return an error message of
the exception.

The exceptions are logged by the commands and their operators. Depending of the
logging level either the whole stack trace is logged or just an error message.

\TheException{CommandException}%
\TypeTable{CommandException & \code{extends} IOException\cite{ioexception13} \\}{%
% constructors
& CommandException(message, cause) \\
& CommandException(message) \\
\midrule
% properties
& context \\
\midrule
% operators
& addContext \\
}

Exception thrown if the command encounters an error while executing.

\TheException{ExecuteException}%
\TypeTable{ExecuteException & \code{extends} CommandException \\}{%
% constructors
& ExecutionException(message, cause, exitValue) \\
& ExecutionException(message, exitValue) \\
\midrule
% properties
& theValue \\
\midrule
% operators
& addContext \\
}

As a convention a command returning an exit value \code{0} returns 
successfully and exit value other then \code{0} indicates that the command
terminated unsuccessfully. If a command returns an exit value other then \code{0}
then a \Exception{ExecuteException} will be thrown.

The successfull exit values can be set to command. So if the command does not
follow the convention or different exit values should be treaded as success,
those values can be set and the exception will not be thrown.

\begin{lstlisting}[style=Groovybash, label={lst:example_execute_exception1}]
try {
  command(arguments)
} catch (ExecuteException e) {
  if (e.theValue == 1) { ... }
  else { ... }
}
\end{lstlisting}

\begin{lstlisting}[style=Groovybash, label={lst:example_execute_exception2}]
try {
  command(arguments)
} catch (ExecuteException e) {
  switch (e.theValue) {
  case 1: ...
  case 2: ...
  default: ...
  }
}
\end{lstlisting}

\TheException[DirectoryNotFoundException]{DirectoryNotFound\\Exception}%
\TypeTable{DirectoryNotFoundException & \code{extends} FileNotFoundException\cite{filenotfoundexception13} \\}{%
% constructors
& DirectoryNotFoundException(directory) \\
\midrule
% properties
& theDirectory \\
\midrule
% operators
& addContext \\
}

Thrown if the specified directory path could not be found or read. 
The \Property{theDirectory} contains the directory\cite{file13}
that was not found.

\begin{lstlisting}[style=Groovybash, label={lst:example_execute_exception1}]
try {
  cd directory
} catch (DirectoryNotFoundException e) {
  // ask for a new directory
}
\end{lstlisting}

