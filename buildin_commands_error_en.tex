\TheCommand{error}%
\Command{error message [, arguments...]}

\TypeTable{error \\}{%
% properties
theContext \\
isEnabled \\
%\midrule
% operators
}

Log a message at the error level of the logger.
The logging facade Slf4j\cite{slf4j13} is used and the used logger is
the global script logger.

\begin{asparadesc}
%
\item[\code{message}] \hfill \\
The message.
%
\item[\code{arguments...}] \hfill \\
The arguments for the message.
%
\item[\code{theContext}] \hfill \\
The name of the logger that is used. Should return the class name of the current
script. The property can only be read.
%
\item[\code{isEnabled}] \hfill \\
Tests if the error level is enabled. Is \code{true} if the error level is enabled;
\code{false} if not. The property can only be read; whether or not the error
level is enabled is configured by the global script logger.
%
\end{asparadesc}

\begin{lstlisting}[style=Groovybash, label={lst:example_error1}, title={%
Outputs a error logging message with arguments.}]
error "The error message."
error "The variable {} should be {}.", variable, expected
error "The variable $variable should be $expected."
\end{lstlisting}

\begin{lstlisting}[style=Groovybash, label={lst:example_error2}, title={%
Prints the name of the current logger.}]
echo error.theContext
\end{lstlisting}

\begin{lstlisting}[style=Groovybash, label={lst:example_error3}, title={%
Tests if the error level is enabled.}]
if (error.isEnabled) {
    error "The error message."
}
\end{lstlisting}

