\TheCommand{block}%

\TypeTable{block \\}{%
% properties
theId \\
theLabel \\
theName \\
theUUID \\
thePath \\
\midrule
% operators
as(\Arg{name}) \\
autoFsck() \\
autoFsck(\Arg{force}) \\
fsck() \\
fsck(\Arg{force}) \\
mount(\Arg{path}) \\
mount(\Arg{mount}, \Arg{path}) \\
umount() \\
umount(\Arg{path}) \\
isMounted(\Arg{path}) \\
size(\Arg{unit}) \\
resize(\Arg{size}, \Arg{unit}) \\
}

The \Command{block} device is a generic device that can be resized, have a file system 
and can be mounted on a directory. That device can be, for example, a hard disk,
a loop-device, an encrypted container.

\subsubsection*{Fields}

\Operator{theId} returns the ID of the block device.

\Operator{theLabel} returns the label of the block device.

\Operator{theName} returns the name of the block device.

\Operator{theUUID} returns the UUID of the block device.

\Operator{thePath} returns the path of the device.

\subsubsection*{Operators}

\Operator{as(\Arg{name})} converts this device to a different type of device.
Set \Arg{name} to the name of the device type to convert to.
Returns this device converted to the new type.
Throws \Exception{CommandException} if there were some error converting the device.

\Operator{fsck()} checks the file system on the device.
Throws \Exception{IOException} if there was an error checking the file system of the device.

\Operator{fsck(\Arg{force})} Checks the file system on the device.
Set \Arg{force} to true to force the check even if the file system is clean.
Throws \Exception{IOException} if there was an error checking the file system of the device.

\Operator{autoFsck()} Checks the file system on the device and automatically repair it.
Throws \Exception{IOException} if there was an error checking the file system of the device.

\Operator{autoFsck(\Arg{force})} Checks the file system on the device. 
Set \Arg{force} to true to force the check even if the file system is clean.
Throws \Exception{IOException} if there was an error checking the file system of the device.

\Operator{mount(\Arg{path})} Mounts the device from the specified path.
Set \Arg{path} to the \TypeFile{} path.
Throws \Exception{NullPointerException} if the specified path is null;
\Exception{IllegalArgumentException} if the specified path is not a directory or if the device is
already mounted on the path. Throws \Exception{IOException} if there was an 
error mounting or unmount the device.

\Operator{mount(\Arg{mount}, \Arg{path})} Mounts the device from a specified path.
Set \Arg{mount} to true if the device should be mounted at the specified
path or to false if the device should be unmounted from the specified path.
Set \Arg{path} to the \TypeFile{} path.
Throws \Exception{NullPointerException} if the specified path is null;
\Exception{IllegalArgumentException} if the specified path is not a directory or if the device is
already mounted on the path or if the device is already mounted on the path or if the device
is not mounted on the path. Throws \Exception{IOException} if there was an 
error mounting or unmount the device.

\Operator{umount()} Unmounts the device from mounted paths.
Throws \Exception{IOException} if there was an 
error mounting or unmount the device.

\Operator{umount(\Arg{path})} Unmounts the device from the specified path.
Set \Arg{path} to the \TypeFile{} path.
Throws \Exception{NullPointerException} if the specified path is null;
\Exception{IllegalArgumentException} if the device is not mounted on the path;
\Exception{IOException} if there was an error unmount the device.

\Operator{isMounted(\Arg{path})} Returns if the device is mounted on the 
specified path. Returns true if the device is mounted on the specified path or
false if not. Set \Arg{path} to the \TypeFile{} path.
Throws \Exception{NullPointerException} if the specified path is null;
\Exception{IOException} if there was an error checking if the device is mounted.

\Operator{size(\Arg{unit})} Returns the size of the device. 
Set \Arg{unit} to the unit of the size, like bytes, blocks or logical 
extents, see \Type*{BlockDeviceUnits}.
Throws \Exception{IOException} if there was an error get the size of the device;
\Exception{UnsupportedOperationException} if the file system of the block device is not supported.

\Operator{resize(\Arg{size}, \Arg{unit})} Resizes the device to the new size.
Set \Arg{size} to the new size in units.
Set \Arg{unit} to the unit of the size, like bytes, blocks or logical 
extents, see \Type*{BlockDeviceUnits}.
Throws \Exception{IOException} if there was an error resizing the device;
\Exception{UnsupportedOperationException} if the file system of the block device is not supported.

\subsubsection*{Variables}

\Operator{BLKID\_COMMAND} the path of the \code{blkid} tool, usually it is found in \code{/usr/sbin/blkid}.

