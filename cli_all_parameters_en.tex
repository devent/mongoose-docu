
\setlength\LTleft{0pt}
\setlength\LTright{0pt}
\begin{longtable}{@{\extracolsep{1em}}p{0.3\textwidth}lp{0.5\textwidth}}
\tbhead{Parameter} & \tbhead{Type} & \tbhead{Description} \\
\toprule
\endhead

%%
\code{-project-type} & \code{<ENUM>} &
\multirow{1}{0.5\columnwidth}{The noise device source type.
\marker\refpar{par:app_project_type}.} \\
\code{-project} & & \\*
\code{-t} & & \\*
\hdashline

%%
\code{-project-output} & \code{<ENUM>} &
\multirow{1}{0.5\columnwidth}{The project output type.
\marker\refpar{par:app_project_output}.} \\
\code{-output} & & \\*
\code{-o} & & \\*

\bottomrule
\caption{Application parameter list}
\end{longtable}

\begin{longtable}{@{\extracolsep{1em}}p{0.3\textwidth}lp{0.5\textwidth}}
\tbhead{Parameter} & \tbhead{Type} & \tbhead{Description} \\
\toprule
\endhead

%%
\code{-Pfile} & \code{<FILE>} &
\multirow{1}{0.5\columnwidth}{The CVS file to load.
\marker\refpar{par:csvfile_file}.} \\
\code{-Pf} & & \\*
\hdashline

%%
\code{-Purl} & \code{<URL>} &
\multirow{1}{0.5\columnwidth}{The URL of the CVS file to load.
\marker\refpar{par:csvfile_file}.} \\
\code{-Pu} & & \\*
\hdashline

%%
\code{-Ppartition-size} & \code{<SIZE>} &
\multirow{2}{0.5\columnwidth}{%
The partition size of the data. \marker\refpar{par:csvfile_partition_size}.} \\
\code{-Psize} & & \\
\code{-Ps} & & \\*
\hdashline

%%
\code{-Pderivations-count} & \code{<SIZE>} &
\multirow{2}{0.5\columnwidth}{%
How many derivations we do with calculate. \marker\refpar{par:csvfile_derivations_count}.} \\
\code{-Pderivations} & & \\
\code{-Pd} & & \\*
\hdashline

%%
\code{-Pcolumns} & \code{<LIST>} &
\multirow{3}{0.5\columnwidth}{%
The number of columns or the sensors that we use from the data.
\marker\refpar{par:csvfile_columns}.} \\
\code{-Pc} & & \\*
& & \\*
\hdashline

%%
\code{-Pvalues-per-column} & \code{<SIZE>} &
\multirow{2}{0.5\columnwidth}{%
How many values we have per column or sensor. \marker\refpar{par:csvfile_values_per_column}.} \\
\code{-Pvalues} & & \\
\code{-Pv} & & \\*
\hdashline

%%
\code{-Pseparator} & \code{<CHAR>} &
\multirow{1}{0.5\columnwidth}{%
The separator character of the CSV file. \marker\refpar{par:csvfile_separator}.} \\
\code{-PS} & & \\*
\hdashline

%%
\code{-Pcharset} & \code{<NAME>} &
\multirow{1}{0.5\columnwidth}{The character set of the CSV file.
\marker\refpar{par:csvfile_charset}.} \\
\code{-PT} & & \\*
\hdashline

%%
\code{-Pheaders} & &
\multirow{2}{0.5\columnwidth}{Flag that specify if the CSV file have a
header row. \marker\refpar{par:csvfile_headers}.} \\
\code{-Ph} & & \\*
\hdashline

%%
\code{-Pdata-calculus-factory} & \code{<NAME>} &
\multirow{4}{0.5\columnwidth}{Sets the data calculus factory that creates
the object that will calculate the derivation or integration of the data.
\marker\refpar{par:csvfile_data_calculus_factory}.} \\
\code{-Pcalculus} & & \\
\code{-PF} & & \\*
& & \\*
\bottomrule

\caption{CSV project parameter list}
\end{longtable}

\begin{longtable}{@{\extracolsep{1em}}p{0.3\textwidth}lp{0.5\textwidth}}
\tbhead{Parameter} & \tbhead{Type} & \tbhead{Description} \\*
\toprule
\endfirsthead
\tbhead{Parameter} & \tbhead{Type} & \tbhead{Description} \\*
\endhead
Continue on next page\dots
\endfoot
\endlastfoot

%%
\code{-Ppartition-size} & \code{<SIZE>} &
\multirow{2}{0.5\columnwidth}{%
The partition size of the data. \marker\refpar{par:crbsnoise_partition_size}.} \\*
\code{-Psize} & & \\*
\code{-Ps} & & \\*
\hdashline

%%
\code{-Pderivations-count} & \code{<SIZE>} &
\multirow{2}{0.5\columnwidth}{%
How many derivations we do with calculate. \marker\refpar{par:crbsnoise_derivations_count}.} \\*
\code{-Pderivations} & & \\*
\code{-Pd} & & \\*
\hdashline

%%
\code{-Psensors} & \code{<LIST>} &
\multirow{3}{0.5\columnwidth}{%
The number of the sensors that we use from the noise device.
\marker\refpar{par:crbsnoise_sensors}.} \\*
\code{-PS} & & \\*
& & \\*
\hdashline

%%
\code{-Pvalues-per-sensor} & \code{<SIZE>} &
\multirow{2}{0.5\columnwidth}{%
How many values we have per sensor. \marker\refpar{par:crbsnoise_values_per_sensor}.} \\*
\code{-Pvalues} & & \\*
\code{-Pv} & & \\*
\hdashline

%%
\code{-Pdata-calculus-factory} & \code{<NAME>} &
\multirow{4}{0.5\columnwidth}{Sets the data calculus factory that creates
the object that will calculate the derivation or integration of the data.
\marker\refpar{par:crbsnoise_data_calculus_factory}.} \\*
\code{-Pcalculus} & & \\*
\code{-PF} & & \\*
& & \\*
\hdashline

%%
\code{-Paddress} & \code{<NUMBER>} &
\multirow{2}{0.5\columnwidth}{%
The internet address of the CRBS noise device. \marker\refpar{par:crbsnoise_address}.} \\*
\code{-Pa} & & \\*
 & & \\*
\hdashline

%%
\code{-Pcommand-port} & \code{<NUMBER>} &
\multirow{2}{0.5\columnwidth}{%
The command port number of the CRBS noise device. \marker\refpar{par:crbsnoise_command_port}.} \\*
\code{-Pcommand} & & \\*
\code{-PC} & & \\*
\hdashline

%%
\code{-Pdata-port} & \code{<NUMBER>} &
\multirow{2}{0.5\columnwidth}{%
The data port number of the CRBS noise device. \marker\refpar{par:crbsnoise_data_port}.} \\*
\code{-Pdata} & & \\*
\code{-PD} & & \\*
\hdashline

%%
\code{-Psensor-type} & \code{<NAME>} &
\multirow{2}{0.5\columnwidth}{%
Set the sensor type to use, \marker\refpar{par:crbsnoise_sensor_type}.} \\*
\code{-Ptype} & & \\*
\code{-Pt} & & \\*
\hdashline

%%
\code{-Psampling-frequency} & \code{<DECIMAL>} &
\multirow{2}{0.5\columnwidth}{%
Set the sampling frequency in Hz to use, \marker\refpar{par:crbsnoise_sampling_frequency}.} \\*
\code{-Pfrequency} & & \\*
\code{-Pf} & & \\*
\hdashline

%%
\code{-Pstabilization-time} & \code{<NUMBER>} &
\multirow{2}{0.5\columnwidth}{%
Set the stabilization time in seconds to use, \marker\refpar{par:crbsnoise_stabilization_time}.} \\*
\code{-Ptime} & & \\*
\code{-PT} & & \\*

\bottomrule

\caption{CRBS noise project parameter list}
\end{longtable}

