\TheCommand{log}%
\Command{log type|object|name}

\TypeTable{log \\}{%
% properties
theContext \\
isEnabled \\
%\midrule
% operators
}

Creates a new logger for the specified type, object or name.
The logging facade Slf4j\cite{slf4j13} is used and the created logger 
is of type \Type{Logger}\cite{logger13}.

\begin{asparadesc}
%
\item[\code{theContext}] \hfill \\
The name of the logger that is used. Should return the class name of the current
script.
%
\item[\code{isEnabled}] \hfill \\
Tests if the info level is enabled. Is \code{true} if the info level is enabled;
\code{false} if not.
%
\end{asparadesc}

\begin{lstlisting}[style=Groovybash, label={lst:example_log}, title={%
Creates a new logger with the context.}]
def mylogger = log this
def mylogger = log "myscript"
def mylogger = log MyClass
mylogger.debug "Debug message for FooClass."
\end{lstlisting}

