\begin{multicols*}{2}[\section{Kenndaten}]

\subsection{Genaue Synchronisation}

Eine möglichst genauso Synchronisation der beiden Instanzen auf den zwei Maschinen
ist nötig um eine möglichst hohe Wahrscheinlichkeit zu haben, dass zwei gleiche Schlüssel
generiert wurden. Diese Genauigkeit muss möglichst an das Ideal mit $\Delta t = 0$,
wobei $\Delta t=|t_a - t_b|$ die Differenz der beiden Zeitpunkte $t_a$ und $t_b$ ist,
bei dem die beiden Instanzen auf Maschinen a und b einen Schlüssel generieren,
so nah wie möglich herankommen.

\subsection{Überprüfbarkeit}

Man muss die Eingabe- und die Ausgabedaten aufzeichnen können. Unter den Eingabedaten
versteht man die rohen Rauschdaten aus den Rauschquellen und unter den Ausgabedaten
die berechneten Schlüssel. Die berechneten Schlüssel müssen einem eindeutigem Zeitpunkt
zugeordnet sein, wann sie generiert wurden. Dies ist nötig, damit wir die generierten
Schlüssel von den zwei Instanzen auf den Maschinen a und b vergleichen können.

\end{multicols*}

