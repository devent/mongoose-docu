%%
\label{par:detox}
\parparameter{\parcommand{detox} \partype{file}}

Remove all ``toxic'' characters from a file name. The \partype{file} argument
can be a \code{File} or the file name. Toxic characters are:
\begin{compactitem}
\item white space characters \code{[ \textbackslash{t}\textbackslash{n}\textbackslash{x0B}\textbackslash{f}\textbackslash{r}]}
\item control characters: \code{[\textbackslash{x00}-\textbackslash{x1F}\textbackslash{x7F}]}
\item punctuation characters that have special meanings in the GNU Bash: \code{["'`]}
\item punctuation characters that are the file and path separator characters on systems: \code{[/\textbackslash;:]}
\end{compactitem}

The name is trimmed on both ends, white space and punctuation characters are replaced with the underscore character \codequoted{\_}
whereas control characters are deleted from the file name. Double underscore characters are replaced with a single
underscore.

\subparagraph{Return Value:}

The return value is an instance of \code{FileValue}. It is the file name converted
to a file after the removal of toxic characters.

\subparagraph{Example:}

\begin{lstlisting}[style=Groovybash, label={lst:example_cd}]
fileName = "  my file with \t\n stuff.txt"
name = detox fileName
\end{lstlisting}

