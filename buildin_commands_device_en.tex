\TheCommand{device}%
\Command{device [type: string] [loop: path] path [args...]}

\TypeTable{device \\}{%
% properties
type \\
device \\
loop \\
theType \\
theDevice \\
theName \\
%\midrule
% operators
}

Creates a device for the specified device \Arg{path}. The \Arg{type}
of the device needs to be set; as an alternative the device \Arg{path}
can be an URI which scheme part is used as the device type.
The \Arg{args...} arguments are passed on to the device.

If the device path is a regular file and the device type is not a loop device, 
then the file is assumed to be an image file for which a loop device is 
first created using the \Command*{loop} command. The loop device will be the
next available loop device if not specified in \Arg{loop}.
After the loop device is setup the specified device type is created.
The \code{file}\cite{mancxfile13} tool is used to identify the file type.

\begin{asparadesc}
%
\item[\code{type}] \hfill \\
the device type.
%
\item[\code{loop}] \hfill \\
the loop path.
%
\item[\code{device}] \hfill \\
the device path.
%
\item[\code{theType}] \hfill \\
returns the device type.
%
\item[\code{theDevice}] \hfill \\
returns the device path.
%
\item[\code{theName}] \hfill \\
returns the name of the command.
%
\item[\code{FILE\_COMMAND}] \hfill \\
the path of the \code{file} tool.
%
\end{asparadesc}

\begin{lstlisting}[style=Groovybash, label={lst:example_device1}, title={
Creates a block device for the device path.}]
def block = device type: "block", "/dev/sda1"
def block = device "block:///dev/sda1"
block.autoFsck()
block.mount "/media/sda1"
\end{lstlisting}

\begin{lstlisting}[style=Groovybash, label={lst:example_device2}, title={
Creates a loop device for the image file.}]
def block = device "block://image.dd"
def block = device loop: "loop1", "block://image.dd"
def block = device type: "block", loop: "loop1", "image.dd"
\end{lstlisting}

\begin{lstlisting}[style=Groovybash, label={lst:example_device3}, title={
Passing of arguments to the created device.}]
def crypt = device type: "crypt", "file://image.dd", "password"
crypt.mount "/media/image"
\end{lstlisting}

