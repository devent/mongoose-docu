\TheCommand{device}%
\Command{device [type: string] [, loop: path] , path [, args...]}

\TypeTable{device \\}{%
% properties
type \\
device \\
loop \\
theType \\
theDevice \\
theName \\
%\midrule
% operators
}

Creates a device for the specified device \Arg{path}. The \Arg{type}
of the device needs to be set; as an alternative the device \Arg{path}
can be an URI which scheme part is used as the device type.
The \Arg{args...} arguments are passed on to the device.

If the device path is a regular file and the device type is not a loop device, 
then the file is assumed to be an image file for which a loop device is 
first created using the \Command*{loop} command. The loop device will be the
next available loop device, if nothing was specified in the \Arg{loop} argument.
After the loop device was created, the specified device type is created.
The \code{file}\cite{mancxfile13} tool is used to identify the file type.

All additional arguments are passed on directly to the device, like passwords,
loop device name, etc. Furthermore, after the device is created, operations
and properties not defined in the command are passed on to the underlying device.

The command will return the device for additional operators on the device
with the property \Operator{theDevice}. The type of the returned device is 
dependend on the type specified:

\begin{asparaitem}
%
\item \qcode{block} see \Command*{block} device;
\item \qcode{crypt} see \Command*{crypt} device;
%\item[\qcode{mdrain}] see \Command*{block} command;
%
\end{asparaitem}


\begin{asparadesc}
%
\item[\code{type}] \hfill \\
the device type.
%
\item[\code{loop}] \hfill \\
the loop path.
%
\item[\code{device}] \hfill \\
the device path.
%
\item[\code{theType}] \hfill \\
returns the device type.
%
\item[\code{theDevice}] \hfill \\
returns the device.
%
\item[\code{theName}] \hfill \\
returns the name of the command.
%
\item[\code{FILE\_COMMAND}] \hfill \\
the path of the \code{file} tool.
%
\end{asparadesc}

\TheCommand{block}%

\TypeTable{block \\}{%
% properties
theId \\
theLabel \\
theName \\
theUUID \\
thePath \\
\midrule
% operators
as(\Arg{name}) \\
autoFsck() \\
autoFsck(\Arg{force}) \\
fsck() \\
fsck(\Arg{force}) \\
mount(\Arg{path}) \\
mount(\Arg{mount}, \Arg{path}) \\
umount() \\
umount(\Arg{path}) \\
isMounted(\Arg{path}) \\
size(\Arg{unit}) \\
resize(\Arg{size}, \Arg{unit}) \\
}

The \Command{block} device is a generic device that can be resized, have a file system 
and can be mounted on a directory. That device can be, for example, a hard disk,
a loop-device, an encrypted container.

\subsubsection*{Fields}

\Operator{theId} returns the ID of the block device.

\Operator{theLabel} returns the label of the block device.

\Operator{theName} returns the name of the block device.

\Operator{theUUID} returns the UUID of the block device.

\Operator{thePath} returns the path of the device.

\subsubsection*{Operators}

\Operator{as(\Arg{name})} converts this device to a different type of device.
Set \Arg{name} to the name of the device type to convert to.
Returns this device converted to the new type.
Throws \Exception{CommandException} if there were some error converting the device.

\Operator{fsck()} checks the file system on the device.
Throws \Exception{IOException} if there was an error checking the file system of the device.

\Operator{fsck(\Arg{force})} Checks the file system on the device.
Set \Arg{force} to true to force the check even if the file system is clean.
Throws \Exception{IOException} if there was an error checking the file system of the device.

\Operator{autoFsck()} Checks the file system on the device and automatically repair it.
Throws \Exception{IOException} if there was an error checking the file system of the device.

\Operator{autoFsck(\Arg{force})} Checks the file system on the device. 
Set \Arg{force} to true to force the check even if the file system is clean.
Throws \Exception{IOException} if there was an error checking the file system of the device.

\Operator{mount(\Arg{path})} Mounts the device from the specified path.
Set \Arg{path} to the \TypeFile{} path.
Throws \Exception{NullPointerException} if the specified path is null;
\Exception{IllegalArgumentException} if the specified path is not a directory or if the device is
already mounted on the path. Throws \Exception{IOException} if there was an 
error mounting or unmount the device.

\Operator{mount(\Arg{mount}, \Arg{path})} Mounts the device from a specified path.
Set \Arg{mount} to true if the device should be mounted at the specified
path or to false if the device should be unmounted from the specified path.
Set \Arg{path} to the \TypeFile{} path.
Throws \Exception{NullPointerException} if the specified path is null;
\Exception{IllegalArgumentException} if the specified path is not a directory or if the device is
already mounted on the path or if the device is already mounted on the path or if the device
is not mounted on the path. Throws \Exception{IOException} if there was an 
error mounting or unmount the device.

\Operator{umount()} Unmounts the device from mounted paths.
Throws \Exception{IOException} if there was an 
error mounting or unmount the device.

\Operator{umount(\Arg{path})} Unmounts the device from the specified path.
Set \Arg{path} to the \TypeFile{} path.
Throws \Exception{NullPointerException} if the specified path is null;
\Exception{IllegalArgumentException} if the device is not mounted on the path;
\Exception{IOException} if there was an error unmount the device.

\Operator{isMounted(\Arg{path})} Returns if the device is mounted on the 
specified path. Returns true if the device is mounted on the specified path or
false if not. Set \Arg{path} to the \TypeFile{} path.
Throws \Exception{NullPointerException} if the specified path is null;
\Exception{IOException} if there was an error checking if the device is mounted.

\Operator{size(\Arg{unit})} Returns the size of the device. 
Set \Arg{unit} to the unit of the size, like bytes, blocks or logical 
extents, see \Type*{BlockDeviceUnits}.
Throws \Exception{IOException} if there was an error get the size of the device;
\Exception{UnsupportedOperationException} if the file system of the block device is not supported.

\Operator{resize(\Arg{size}, \Arg{unit})} Resizes the device to the new size.
Set \Arg{size} to the new size in units.
Set \Arg{unit} to the unit of the size, like bytes, blocks or logical 
extents, see \Type*{BlockDeviceUnits}.
Throws \Exception{IOException} if there was an error resizing the device;
\Exception{UnsupportedOperationException} if the file system of the block device is not supported.

\subsubsection*{Variables}

\Operator{BLKID\_COMMAND} the path of the \code{blkid} tool, usually it is found in \code{/usr/sbin/blkid}.


\TheCommand{crypt}%

\TypeTable{crypt \\}{%
% properties
theLabel \\
theName \\
theUUID \\
thePath \\
\midrule
% operators
open(\Arg{args}, \Arg{name}) \\
close(\Arg{name}) \\
isOpen(\Arg{name}) \\
as(\Arg{name}) \\
autoFsck() \\
autoFsck(\Arg{force}) \\
fsck() \\
fsck(\Arg{force}) \\
mount(\Arg{path}) \\
mount(\Arg{mount}, \Arg{path}) \\
umount() \\
umount(\Arg{path}) \\
isMounted(\Arg{path}) \\
size(\Arg{unit}) \\
resize(\Arg{size}, \Arg{unit}) \\
}

The \Command{crypt} device is an encrypted container with or without a
file system.

\subsubsection*{Fields}

The fields 
\begin{inparaitem}
\item \Operator{theLabel}
\item \Operator{theName}
\item \Operator{theUUID}
\item \Operator{thePath}
\end{inparaitem}
are returning the same information as from the \Command*{block} device.

\subsubsection*{Operators}

\Operator{open(\Arg{args}, \Arg{name})} 
opens the encrypted device under the specified name.
Set \Arg{name} to the name of the opened device.
Set \Arg{args} to pass additional arguments, like the passport or passphrase.
Throws \Exception{IOException} if there was an error opening the device.

For a LUKS device the passphrase is read from the standard input of the command
if the passphrase is not explicetly specified as the \Arg{passphrase} argument.

\Operator{close(\Arg{name})} 
closes the encrypted device under the specified name.
Set \Arg{name} to the name of the opened device.
Throws \Exception{IOException} if there was an error closing the device.

\Operator{isOpen(\Arg{name})} 
checks if the encrypted device is open under the specified name.
Set \Arg{name} to the name of the opened device.
Returns true if the device is open under the specified name.
Throws \Exception{IOException} if there was an error checking the device.

The operators 
\begin{inparaitem}
\item \Operator{as(\Arg{name})}
\item \Operator{autoFsck()}
\item \Operator{autoFsck(\Arg{force})}
\item \Operator{fsck()}
\item \Operator{fsck(\Arg{force})}
\item \Operator{mount\\(\Arg{path})}
\item \Operator{mount(\Arg{mount}, \Arg{path})}
\item \Operator{umount()}
\item \Operator{umount(\Arg{path})}
\item \Operator{isMounted(\Arg{path})}
\item \Operator{size(\Arg{unit})}
\item \Operator{resize(\Arg{size}, \Arg{unit})}
\end{inparaitem}
are behaving the same as the operators from the \Command*{block} device,
with the exception that the encrypted device must be open.

\subsubsection*{Variables}

\Operator{BLKID\_COMMAND} the path of the \code{blkid} tool, usually it is 
found in \code{/usr/sbin/blkid}.

\Operator{CRYPTSETUP\_COMMAND} the path of the \code{cryptsetup} tool, usually it is 
found in \code{/usr/sbin/cryptsetup}.



\begin{lstlisting}[style=Groovybash, label={lst:example_device1}, title={
Creates a block device for the device path.}]
def device = device type: "block", "/dev/sda1"// or
def device = device "block:///dev/sda1"
def block = device.theDevice
block.autoFsck()
block.mount "/media/sda1"
// alternative
def device = device "block:///dev/sda1"
device.autoFsck()
device.mount "/media/sda1"
\end{lstlisting}

\begin{lstlisting}[style=Groovybash, label={lst:example_device2}, title={
Creates a block device for the device path and apply operations on it, short syntax}]
device "block:///dev/sda1" autoFsck()
device "block:///dev/sda1" mount "/media/sda1"
\end{lstlisting}

\begin{lstlisting}[style=Groovybash, label={lst:example_device2}, title={
Creates a loop device for the image file.}]
def block = device "block://image.dd" theDevice
def block = device loop: "loop1", "block://image.dd" theDevice
def block = device type: "block", loop: "loop1", "image.dd" theDevice
\end{lstlisting}

\begin{lstlisting}[style=Groovybash, label={lst:example_device3}, title={
Passing of arguments to the created device.}]
def crypt = device type: "crypt", "file://image.dd", "password" theDevice
crypt.mount "/media/image"
\end{lstlisting}
