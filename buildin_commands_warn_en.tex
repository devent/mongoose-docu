\TheCommand{warn}%
\Command{warn message [, args]}

\TypeTable{warn \\}{%
% properties
theContext \\
isEnabled \\
%\midrule
% operators
}

Log a message at the warn level of the logger.
The logging facade Slf4j\cite{slf4j13} is used and the used logger is
the global script logger.

\begin{asparadesc}
%
\item[\code{message}] \hfill \\
The message.
%
\item[\code{args}] \hfill \\
The arguments for the message.
%
\item[\code{theContext}] \hfill \\
The name of the logger that is used. Should return the class name of the current
script. The property can only be read.
%
\item[\code{isEnabled}] \hfill \\
Tests if the warn level is enabled. Is \code{true} if the warn level is enabled;
\code{false} if not. The property can only be read; whether or not the warn
level is enabled is configured by the global script logger.
%
\end{asparadesc}

\begin{lstlisting}[style=Groovybash, label={lst:example_warn1}, title={%
Outputs a warn logging message with arguments.}]
warn "The warn message."
warn "The variable {} should be {}.", variable, expected
warn "The variable $variable should be $expected."
\end{lstlisting}

\begin{lstlisting}[style=Groovybash, label={lst:example_warn2}, title={%
Prints the name of the current logger.}]
echo warn.theContext
\end{lstlisting}

\begin{lstlisting}[style=Groovybash, label={lst:example_warn3}, title={%
Tests if the warn level is enabled.}]
if (warn.isEnabled) {
    warn "The warn message."
}
\end{lstlisting}

