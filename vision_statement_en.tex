\section{Vision Statement}

The software is to create bash-like scripts in the Groovy language.

The current bash-scripts are very powerful but also limited in some ways.
The bash-scripts have the following major features: a) all commands that are not
keywords or operators are treated as build-in or external commands and are
executed, thus making a command an intergrated part in the script, like
functions; b) the input and output streams of the commands can be redirected. The
output of one command can be used as the input of a second command, thus making
command chaining possible; c) the ``sha-bang''\footnote{\code{\#!/bin/bash}} line 
will define the shell that will load
and execute the bash-script and make it possible to start the bash-script like an
executable file.

But the bash-scripts have also disadvantages. Most notably there are no 
explicit parameter, no classes and no exceptions in bash-scripts. Programming 
is done procedural and parameters are passed in special 
variables\footnote{\code{\$1\dots\$9,\$*,\$?}}. Failures are communicated with
return codes that can be accessed with a global variable\footnote{\code{\$?}}.
This makes writing robust scripts difficult.

The Groovy language can enhance a script with 
object oriented programming principles, like classes, objects and exceptions.
Groovy is a dynamic programming language and as such we can create a domain 
specific language (DSL) designed for bash-like scripts.
