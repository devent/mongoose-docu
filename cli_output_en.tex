\section{Command Line Output}

With the \code{-project-output}, \code{-output} or \code{-o <NAME>|<LIST>}
parameter we can specify the output of the data, \marker\refpar{par:app_project_output}.

The supported outputs are:
\begin{asparadesc}
\item[\code{output:console}]
Outputs the data in the console.
\item[\code{output:noise}]
Outputs the not modified noise data.
\item[\code{output:derivated}]
Outputs the calculated derivations.
\item[\code{output:fractal}]
Outputs the calculated fractal dimensions in the console.
\item[\code{output:projects}]
Prints the available projects and output types.
\end{asparadesc}

\subsection{output:console}

The data will be printed in the console. That is useful for small amount of data
to see the output.

The output of the console can also be redirected to a file or to a different
application.

\begin{lstlisting}[style=nonumbers,language=bash, caption={Example redirect to file}]
./java fdsanalysis.jar > output_file.csv
\end{lstlisting}

\subsection{output:noise}

Outputs the noise data without any calculations. This is to read the noise data
from the source and output it on the console.
The first line is a header line that contains the names of the columns.
The header line is only printed for the first partition.
The partitions and the rows are following.

The columns are:
\begin{asparadesc}
\item[partition]
The partition number of the data.
\item[row]
The row number of the data.
\item[columns]
The user columns or sensors.
\item[column 0\dots{n}]
The column $0\dots{n}$ with the data.
\end{asparadesc}

The \reflst{lst:data_noise} shows the noise data output for the
first partition and noise diodes with one value per column.
The \reflst{lst:data_gmr} shows the noise data output for the
first partition and GMR data with GMR-X and GMR-Y values per column.

\subsection{output:derivated}

Outputs the derivations in a Comma Separated Values format. The first line is
a header line that contains the names of the columns.
The header line is only printed for the first partition.
The partitions and the
rows are following.

The columns are:
\begin{asparadesc}
\item[partition]
The partition number of the data.
\item[row]
The row number of the data.
\item[columns]
The user columns or sensors.
\item[column 0\dots{n}]
The column $0\dots{n}$ with the data.
\end{asparadesc}

The \reflst{lst:derivated_data_noise} shows the derivated data output for the
first partition and noise diodes with one value per column.
The \reflst{lst:derivated_data_gmr} shows the derivated data output for the
first partition and GMR data with GMR-X and GMR-Y values per column.

\subsection{output:fractal}

Outputs the fractal dimensions in a Comma Separated Values format. The first line is
a header line that contains the names of the columns.
The header line is only printed for the first partition.
The partitions and the
rows are following. The fractal data have only one row per partition, the last
row. The values are not the fractal dimensions but the $N$ value from
\refeq{eq:fractal_dimension}. With $k$ is the partition size, the value of $D$
can be calculated.

\begin{equation}\label{eq:fractal_dimension}
D=\frac{\log{k}}{\log{N}}
\end{equation}


The columns are:
\begin{asparadesc}
\item[partition]
The partition number of the data.
\item[row]
The row number of the data.
\item[columns]
The user columns or sensors.
\item[column 0\dots{n}]
The column $0\dots{n}$ with the data.
\end{asparadesc}

The \reflst{lst:fractal_data_noise} shows the fractal data output for the
first partition and noise diodes with one value per column.
The \reflst{lst:fractal_data_gmr} shows the fractal data output for the
first partition and GMR data with GMR-X and GMR-Y values per column.

\subsection{output:projects}

