\subsection{Stream Redirection}

All commands can have file descriptor attached for input and output.
Three file descriptor are assigned to the standard input and output streams
of the command:
\begin{compactitem}
\item file descriptor 0. to standard input;
\item file descriptor 1. to standard output;
\item file descriptor 2. to standard output or errors;
\end{compactitem}

Those file descriptors can be redirected to a different stream
or file. The operators \code{input(), output(), error()} and
operators \code{<<, >>} can redirect the standard outputs
or standard input to a different stream or file. The operator \code{|} 
can pipe the standard output of one command to the standard input of another command.

\TheOperator{input()}%
\Operator{input(object)}

The \Operator{input()} operator redirects the standard input of
a command to the object specified on the right side. The object on the right 
side can be of type \TypeStream or of type \TypeFile or can be a \TypeGenericObject. 
The result of the operator will be the finished command.

The \TypeStream will be read until the end of the stream
is reached; the \TypeFile will be open and the content of the file is read;
A \TypeGenericObject is converted to a string and the string is used as a file name. 
The file with the file name is open and the content of the file is read.

\begin{lstlisting}[style=Groovybash, label={lst:example_input1_op}]
echo.input = stream
echo.input = file
echo.input = "file.txt"
echo.input stream
\end{lstlisting}

\TheOperator{output()}%
\Operator{output([descriptor,] object [, append])}

The \Operator{output()} operator redirects the specified file descriptor of
a command to the object specified on the right side. If the file descriptor
is not set, the file descriptor 1. is used and redirects the standard output, 
The object can be of type \TypeStream or of type \TypeFile or can be a \TypeGenericObject. 

The third argument is a boolean flag that specifies if the stream or file should
be overridden or appended. If set to \code{true} then the stream or file
is appended instead of overridden. If the flag is not set then the override
mode is active.
The result of the operator will be the finished command.

For a \TypeStream the output will be written in the stream; 
for a \TypeFile the file will be open and the output will be written to the file;
A \TypeGenericObject is converted to a string and the string is used as a file name. 
The file with the file name will be open and the output will be written to the file.

\begin{lstlisting}[style=Groovybash, label={lst:example_output1_op}]
echo.output 1, file, true
echo.output 1, stream
echo.output 1, file
echo.output 1, "file.txt"
echo.output = stream
echo.output = file
echo.output = "file.txt"
\end{lstlisting}

\TheOperator{error()}%
\Operator{error(object [, append])}

The \Operator{error()} operator redirects the standard error output stream of
a command to the object specified on the right side, like the \Operator{output} operator.

The second argument is a boolean flag that specifies if the stream or file should
be overridden or appended. If set to \code{true} then the stream or file
is appended instead of overridden. If the flag is not set then the override
mode is active.

\begin{lstlisting}[style=Groovybash, label={lst:example_error_op}]
echo.error file, true
echo.error = stream
echo.error = file
echo.error = "file.txt"
\end{lstlisting}

\TheOperator{<< operator}%
\Operator{command << object}

The \Operator{<<} redirects the standard input stream of the left to 
the object specified on the right side, like the \Operator{input}, but
files and streams are appended instead of overridden if possible. 
The result of the operator will be the finished command.

\TheOperator{>> operator}%
\Operator{command >> object}

The \Operator{>>} redirects the standard output stream of
the left to the object specified on the right side, like the \Operator{output}, but
files and streams are appended instead of overridden.
The result of the operator will be the finished command.

\TheOperator{| operator}%
\Operator{command | command}

The \Operator{|} creates a pipe between the left command and the right command.
It is possible to chain multiple commands together. The result value will be the
last command executed.

\begin{lstlisting}[style=Groovybash, label={lst:example_pipe1_op}]
cat.args(file) | column | echo
\end{lstlisting}

\TheOperator{pipe()}%
\Operator{pipe(command)}

The \Operator{pipe()} operator creates a pipe between the left command 
and the right command, like the \Operator{|} operator. The result value will be the
finished last command.

\begin{lstlisting}[style=Groovybash, label={lst:example_pipe2_op}]
cat.args(file).pipe column pipe echo
\end{lstlisting}

