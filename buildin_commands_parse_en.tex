\TheCommand{parse}%
\Command{parse [arguments,] [valid,] [notValid,] parameter}

\TypeTable{parse \\}{%
% properties
arguments \\
valid \\
notValid \\
theParameter \\
\midrule
% operators
printExample() \\
printSingleLineUsage() \\
printUsage() \\
}

Parse the specified command line parameter. If no 
arguments are specified the command line arguments of the script, stored in
the \Variable{ARGS} build-in variable, are used.
An exception will be thrown if there was an error parsing the command line arguments
by the specified parameter.

The specified parameter is a Java bean object annotated with 
Args4j\cite{args4j13} annotations:
\begin{inparaitem}
\item \Type{Argument};
\item \Type{Option};
\end{inparaitem}

\begin{asparadesc}
%
\item[\code{arguments: array|list}] \hfill \\
The command line arguments. Defaults to the the \Variable{ARGS} build-in variable.
%
\item[\code{valid: closure}] \hfill \\
Sets the closure that is called if the command line arguments are parsed without
any errors. The first argument of the closure will be the parsed parameter.
%
\item[\code{notValid: closure}] \hfill \\
Sets the closure that is called if the command line arguments are parsed with
errors.
%
\item[\code{theParameter}] \hfill \\
The parameter Java bean object. After the command line arguments are parsed
the annotated fields contain the parsed values. The property can only be read.
%
\item[\code{printExample(exampleMode [, resourceBundle])}] \hfill \\
Returns an example of the command line arguments.
\begin{compactdesc}
%
\item[\code{exampleMode}]
the \Type{ExampleMode}\cite{examplemode13} to use. Can be:
\begin{compactitem}
\item \code{ExampleMode.ALL} to return all defined options in the example; 
\item \code{ExampleMode.REQUIRED} to return only the required options in the example; 
\end{compactitem}
%
\item[\code{resourceBundle}]
the \Type{ResourceBundle}\cite{resourcebundle13} to use, can be \code{null}.
%
\end{compactdesc}
%
\item[\code{printSingleLineUsage([resourceBundle])}] \hfill \\
Returns an example of the usage in a single line.
\begin{compactdesc}
%
\item[\code{resourceBundle}]
the \Type{ResourceBundle}\cite{resourcebundle13} to use, can be \code{null}.
%
\end{compactdesc}
%
\item[\code{printUsage([resourceBundle])}] \hfill \\
Returns an example of the usage.
\begin{compactdesc}
%
\item[\code{resourceBundle}]
the \Type{ResourceBundle}\cite{resourcebundle13} to use, can be \code{null}.
%
\end{compactdesc}
%
\end{asparadesc}

\begin{lstlisting}[style=Groovybash, label={lst:example_arguments_parse1}, title={%
Storing the parse command in a variable and set 
the arguments and parse the command line arguments at a later point.}]
def parser = parse.args new Parameter()
parser.valid = { convertFiles(it) }
parser.notValid = { printHelp() }
parser()
\end{lstlisting}

\begin{lstlisting}[style=Groovybash, label={lst:example_arguments_parse2}, title={%
Parse command line arguments; valid and not-valid 
closures are called depending on whether the arguments are valid.}]
parse valid: { convertFiles(it) },
    notValid: { printHelp() },
    new Parameter()
    
def printHelp() {
}

class Parameter {

    @Option(name = "-format")
    String format

    @Argument(required = true, metaVar = "FILES")
    List arguments
}

\end{lstlisting}

\begin{lstlisting}[style=Groovybash, label={lst:example_arguments_parse3}, title={%
Print example and usage.}]
def parser = parse.args new Parameter()
    
def printHelp() {
    echo parser.printExample(ExampleMode.ALL)
    echo parser.printSingleLineUsage()
    echo parser.printUsage()
}

class Parameter {

    @Option(name = "-format")
    String format

    @Argument(required = true, metaVar = "FILES")
    List arguments
}

\end{lstlisting}

