\TheCommand{parse}%
\Command{parse [arguments,] [valid,] [notValid,] parameter}

\TypeTable{parse & implements Command \\}{%
% constructors
\midrule
% properties
& theParameter \\
& arguments \\
& valid \\
& notValid \\
\midrule
% operators
}

Parse the specified command line parameter. If no 
arguments are specified the command line arguments of the script, stored in
the \Variable{ARGS} build-in variable, are used.
An exception will be thrown if there was an error parsing the command line arguments
by the specified parameter.

The specified parameter is a Java bean object annotated with 
Args4j\cite{args4j13} annotations:
\begin{inparaitem}
\item \Type{Argument};
\item \Type{Option};
\end{inparaitem}

\paragraph{Arguments}

\begin{asparadesc}
%
\item[\code{arguments: array|list}] \hfill \\
The command line arguments. Defaults to the the \Variable{ARGS} build-in variable.
%
\item[\code{valid: closure}] \hfill \\
Sets the closure that is called if the command line arguments are parsed without
any errors. The first argument of the closure will be the parsed parameter.
%
\item[\code{notValid: closure}] \hfill \\
Sets the closure that is called if the command line arguments are parsed with
errors.
%
\item[\code{parameter}] \hfill \\
Sets the parameter Java bean object. After the command line arguments are parsed
the annotated fields contain the parsed values.
%
\end{asparadesc}

\paragraph{Properties}

\begin{asparadesc}
%
\item[\code{theParameter}] \hfill \\
The parameter Java bean object. After the command line arguments are parsed
the annotated fields contain the parsed values.
%
\end{asparadesc}

\begin{lstlisting}[style=Groovybash, label={lst:example_arguments_parse}, title={\Command{parse} command: storing the parse command in a variable and set the arguments and parse the command line arguments at a later point.}]
def parser = builtin parse
parser.valid = { convertFiles(it) }
parser.notValid = { printHelp() }
parser.parameter = new Parameter()
parser()
\end{lstlisting}

\begin{lstlisting}[style=Groovybash, label={lst:example_arguments_parse}, title={\Command{parse} command: parse command line arguments; valid and not-valid closures are called depending on whether the arguments are valid.}]
parse valid: { convertFiles(it) },
    notValid: { printHelp() },
    new Parameter()
    
def printHelp() {
}

class Parameter {

    @Option(name = "-format")
    String format

    @Argument(required = true, metaVar = "FILES")
    List arguments
}

\end{lstlisting}

