\TheCommand{info}%
\Command{info message [, arguments...]}

\TypeTable{info \\}{%
% properties
theContext \\
isEnabled \\
%\midrule
% operators
}

Log a message at the info level of the logger.
The logging facade Slf4j\cite{slf4j13} is used and the used logger is
the global script logger.

\begin{asparadesc}
%
\item[\code{message}] \hfill \\
The message.
%
\item[\code{arguments...}] \hfill \\
The arguments for the message.
%
\item[\code{theContext}] \hfill \\
The name of the logger that is used. Should return the class name of the current
script. The property can only be read.
%
\item[\code{isEnabled}] \hfill \\
Tests if the info level is enabled. Is \code{true} if the info level is enabled;
\code{false} if not.  The property can only be read; whether or not the info
level is enabled is configured by the global script logger.
%
\end{asparadesc}

\begin{lstlisting}[style=Groovybash, label={lst:example_info1}, title={%
Outputs a info logging message with arguments.}]
info "The info message."
info "The variable {} should be {}.", variable, expected
info "The variable $variable should be $expected."
\end{lstlisting}

\begin{lstlisting}[style=Groovybash, label={lst:example_info2}, title={%
Prints the name of the current logger.}]
echo info.theContext
\end{lstlisting}

\begin{lstlisting}[style=Groovybash, label={lst:example_info3}, title={%
Tests if the info level is enabled.}]
if (info.isEnabled) {
    info "The info message."
}
\end{lstlisting}

