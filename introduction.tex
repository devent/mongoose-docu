\begin{multicols*}{2}[\chapter{Einleitung}]
 
Die Software soll eine Kommandozeilen-Äquivalenz zu der GUI Software sein. Es
soll alle Möglichkeiten der Schlüsselerzeugung bieten, aber über einen
Kommandozeilen-Schnittstelle. Eine Kommandozeilen-Schnittstelle bietet der
graphischen Schnittstelle entscheidende Vorteile der Automatisierbarkeit und
Überprüfbarkeit.

Es werden immer zwei Instanzen der Anwendung auf zwei verschiedenen Maschinen
benötigt, die jeweils eine Rauschquelle gleichzeitig auslesen können. Diese
Rauschdaten werden nach Global Scaling analysiert und daraus Schlüssel generiert.
Diese Schlüssel sollten bei gleichzeitiger Generierung eine hohe Wahrscheinlichkeit
aufweisen, dass sie identisch sind.

Die zwei wichtigsten Aspekte der Anwendung ist die Korrektheit der Synchronisation
und der Aufzeichnung der generierten Schlüssel zur späten Auswertung.

Die Anwendung soll die folgenden Parameter auswerten können. Zum einen die
Rauschquelle muss auswählbar und konfigurierbar sein. Als Rauschquelle soll vor
allem
\begin{inparaenum}[a)]
\item die Greybox und
\item eine Datei mit vorher aufgezeichnete Rauschdaten
\end{inparaenum}
dienen.

Die Zeitquelle, die zur Synchronisation benötigt wird, soll auswählbar und
konfigurierbar sein. Als Zeitquellen sollen vor allem
\begin{inparaenum}[a)]
\item die Systemzeit des Computers und
\item ein NTP-Server
\end{inparaenum}
dienen.

Die Parameter der Schlüsselerzeugung müssen übergebbar sein, wie
\begin{inparaenum}[a)]
\item die benutzten Sensoren,
\item die Anzahl an Ableitungen,
\item die Parameter der fraktalen Berechnung,
\item die Größe des Schlüssels und
\item die Anzahl der Schlüssel.
\end{inparaenum}

Die Form der Ausgabe soll einstellbar sein. Es sollen vor allem die Ausgabe als 
\begin{inparaenum}[a)]
\item eine CSV Datei
\end{inparaenum}
verfügbar sein.

\end{multicols*}

\begin{multicols*}{2}[\section{Kenndaten}]

\subsection{Genaue Synchronisation}

Eine möglichst genauso Synchronisation der beiden Instanzen auf den zwei Maschinen
ist nötig um eine möglichst hohe Wahrscheinlichkeit zu haben, dass zwei gleiche Schlüssel
generiert wurden. Diese Genauigkeit muss möglichst an das Ideal mit $\Delta t = 0$,
wobei $\Delta t=|t_a - t_b|$ die Differenz der beiden Zeitpunkte $t_a$ und $t_b$ ist,
bei dem die beiden Instanzen auf Maschinen a und b einen Schlüssel generieren,
so nah wie möglich herankommen.

\subsection{Überprüfbarkeit}

Man muss die Eingabe- und die Ausgabedaten aufzeichnen können. Unter den Eingabedaten
versteht man die rohen Rauschdaten aus den Rauschquellen und unter den Ausgabedaten
die berechneten Schlüssel. Die berechneten Schlüssel müssen einem eindeutigem Zeitpunkt
zugeordnet sein, wann sie generiert wurden. Dies ist nötig, damit wir die generierten
Schlüssel von den zwei Instanzen auf den Maschinen a und b vergleichen können.

\end{multicols*}


\section{Design}

\begin{figure}
\includegraphics{high_level_arch} 
\end{figure}


