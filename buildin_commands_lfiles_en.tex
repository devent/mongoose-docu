\TheCommand{listFiles}%
\Command{listFiles [recursive, depth, includeSubDirectories,\\ filter, filterType] [files]}

\TypeTable{listFiles \\}{%
% properties
recursive \\
depth \\
includeSubDirectories \\
filter \\
filterType \\
theFiles \\
%\midrule
% operators
}

Search all files with the specified file names. The file names can contain 
directory listings and wildcats \code{``*?''} or regular expressions.
If no file names are specified, as default all files are returned in the 
current working directory.

\begin{asparadesc}
%
\item[\code{recursive: true|false}] \hfill \\
if \code{true} then the files are searched in sub-directories recursive;
if \code{false} then no sub-directories will be searched. Defaults to \code{false}.
%
\item[\code{depth: (+)number|-1}] \hfill \\
Sets the number of levels of the sub-directories that should be searched.
Set to zero to deactivate recursive search of sub-directories; set to a positive
number to activate recursive search of sub-directories.
Defaults to \code{-1} with means unlimited levels.
%
\item[\code{includeSubDirectories: true|false}] \hfill \\
if \code{true} then the sub-directories of a recursive search are included.
Defaults to \code{false}.
%
\item[\code{filter: FileFilter|closure}] \hfill \\
Sets the \Type{FileFilter}\cite{filefilter13} to filter the found files.
See also the sub-classes of \Type{AbstractFileFilter}\cite{abstractfilefilter13}.
Can be a closure that returns a \code{boolean} value. The closure will be given
the file\cite{file13} as the first argument.
%
\item[\code{filterType: class}] \hfill \\
Sets a filter type that is used as the filter for files. The filter 
must be of type \Type{FileFilter}\cite{filefilter13} and must have
a public constructor that accepts a string as the file name. Defaults
to the \Type{WildcardFileFilter}\cite{wildcardfilefilter13}. Possible canditates
are:
\begin{compactitem}
\item \Type{NameFileFilter}
\item \Type{PrefixFileFilter}
\item \Type{RegexFileFilter}
\item \Type{SuffixFileFilter}
\item \Type{WildcardFileFilter}
\end{compactitem}
%
\item[\code{theFiles}] \hfill \\
The files that are found in the directories. The property can only be read and
is available after the command have finished executing.
%
\end{asparadesc}

\begin{lstlisting}[style=Groovybash, label={lst:example_listFiles_simple}, title={%
List files in the current working directory.}]
def files = listFiles().theFiles
def files = listFiles "*.txt" theFiles
\end{lstlisting}

\begin{lstlisting}[style=Groovybash, label={lst:example_listFiles_arguments}, title={%
List files in the current working directory with some additional arguments.}]
def files = listFiles recursive: true, depth: 5, "*.txt" theFiles
def files = listFiles depth: 5, "*.txt" theFiles
\end{lstlisting}

\begin{lstlisting}[style=Groovybash, label={lst:example_listFiles_iterate}, title={%
Iterate over the found files.}]
listFiles() theFiles.each {
    echo it
}
listFiles "*.jpg" theFiles.each {
    echo it
}
\end{lstlisting}

\begin{lstlisting}[style=Groovybash, label={lst:example_listFiles_filter}, title={%
List files using a custom file filter.}]
def myfilter = [accept: { it.name =~ /.*txt/ }] as FileFilter
def files = listFiles filter: myfilter, "path/to/files"
def files = listFiles filterType: RegexFileFilter, "path/to/files/[abc].*"
listFiles filter: { it.name =~ /.*txt/ } theFiles.each {
    echo it
}
\end{lstlisting}

