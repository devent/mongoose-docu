\TheCommand{crypt}%

\TypeTable{crypt \\}{%
% properties
theLabel \\
theName \\
theUUID \\
thePath \\
\midrule
% operators
open(\Arg{args}, \Arg{name}) \\
close(\Arg{name}) \\
isOpen(\Arg{name}) \\
as(\Arg{name}) \\
autoFsck() \\
autoFsck(\Arg{force}) \\
fsck() \\
fsck(\Arg{force}) \\
mount(\Arg{path}) \\
mount(\Arg{mount}, \Arg{path}) \\
umount() \\
umount(\Arg{path}) \\
isMounted(\Arg{path}) \\
size(\Arg{unit}) \\
resize(\Arg{size}, \Arg{unit}) \\
}

The \Command{crypt} device is an encrypted container with or without a
file system.

\subsubsection*{Fields}

The fields 
\begin{inparaitem}
\item \Operator{theLabel}
\item \Operator{theName}
\item \Operator{theUUID}
\item \Operator{thePath}
\end{inparaitem}
are returning the same information as from the \Command*{block} device.

\subsubsection*{Operators}

\Operator{open(\Arg{args}, \Arg{name})} 
opens the encrypted device under the specified name.
Set \Arg{name} to the name of the opened device.
Set \Arg{args} to pass additional arguments, like the passport or passphrase.
Throws \Exception{IOException} if there was an error opening the device.

For a LUKS device the passphrase is read from the standard input of the command
if the passphrase is not explicetly specified as the \Arg{passphrase} argument.

\Operator{close(\Arg{name})} 
closes the encrypted device under the specified name.
Set \Arg{name} to the name of the opened device.
Throws \Exception{IOException} if there was an error closing the device.

\Operator{isOpen(\Arg{name})} 
checks if the encrypted device is open under the specified name.
Set \Arg{name} to the name of the opened device.
Returns true if the device is open under the specified name.
Throws \Exception{IOException} if there was an error checking the device.

The operators 
\begin{inparaitem}
\item \Operator{as(\Arg{name})}
\item \Operator{autoFsck()}
\item \Operator{autoFsck(\Arg{force})}
\item \Operator{fsck()}
\item \Operator{fsck(\Arg{force})}
\item \Operator{mount\\(\Arg{path})}
\item \Operator{mount(\Arg{mount}, \Arg{path})}
\item \Operator{umount()}
\item \Operator{umount(\Arg{path})}
\item \Operator{isMounted(\Arg{path})}
\item \Operator{size(\Arg{unit})}
\item \Operator{resize(\Arg{size}, \Arg{unit})}
\end{inparaitem}
are behaving the same as the operators from the \Command*{block} device,
with the exception that the encrypted device must be open.

\subsubsection*{Variables}

\Operator{BLKID\_COMMAND} the path of the \code{blkid} tool, usually it is 
found in \code{/usr/sbin/blkid}.

\Operator{CRYPTSETUP\_COMMAND} the path of the \code{cryptsetup} tool, usually it is 
found in \code{/usr/sbin/cryptsetup}.

