\subsection{Executing} 

\TheOperator{call()}%
\Operator{call(object...)}

The \Operator{call()} operator sets the arguments for the command and executes the
command. The script will wait for the command to finish and the return value 
is the finished command.
The arguments are optional and can be a single object, a list of objects,
a map of objects or named arguments. 

External commands have no concept of 
types and all arguments are converted
to a string and passed to the command as command line arguments.
List of arguments, maps and named arguments are passed to the command
in the order they are given.

\begin{lstlisting}[style=Groovybash, label={lst:example_call_op}]
echo()
cat file
cat "-n", file
cat "-n": file
command one: foo, two: bar
\end{lstlisting}

\TheOperator{background()}%
\Operator{background(object...)}

The \Operator{background()} operator sets the arguments for the command and executes the
command in the background. 
The operator can set the arguments for the command like the \Operator{call()}
operator. The return value is a \Type{Future}\cite{future13} of the command.

\begin{lstlisting}[style=Groovybash, label={lst:example_call_op}]
echo.background()
cat.background file
cat.background "-n", file
cat.background "-n": file
command.background one: foo, two: bar
\end{lstlisting}


\TheOperator{args()}%
\Operator{args(object...)}

The \Operator{args()} operator sets the arguments for the command, like 
the \Operator{call()} operator, but will not execute the command. The return
value is the command.

\begin{lstlisting}[style=Groovybash, label={lst:example_args_op}]
cat.args(file)
cat.args "-n", file
cat.args "-n": file
command.args one: foo, two: bar
\end{lstlisting}

