\subsection{Executing} 

\TheOperator{call()}%
\Operator{call(object...)}

The \Operator{call()} operator sets the arguments for the command and executes 
the command. The script will wait for the command to finish and the return 
value 
is the finished command. The arguments are optional and can be a single object, 
a list of objects, a map of objects or named arguments. 

External commands have no concept of 
types and all arguments are converted
to a string and passed to the command as command line arguments.
List of arguments, maps and named arguments are passed to the command
in the order they are given.

\begin{lstlisting}[style=Groovybash, label={lst:example_call_op}, 
title={Execute command with no arguments or with specified arguments either 
direct or by key=value pairs.}]
echo()
cat file
cat "-n", file
cat "-n": file
command one: foo, two: bar
\end{lstlisting}

\TheOperator{background()}%
\Operator{background([listener: object] [listeners: list] object...)}

The \Operator{background()} operator sets the arguments for the command and 
executes the
command in the background. 
The operator can set the arguments for the command like the \Operator{call()}
operator. The return value is a \Type{ListenableFuture}\Type*{ListenableFuture} of the command.

\begin{asparadesc}
%
\item[\code{listener}] \hfill \\
optional, sets the \Type{PropertyChangeListener}\cite{propertychangelistener13}
listener to be informed when the command finishes.
%
\item[\code{listeners}] \hfill \\
optional, sets a list of \Type{PropertyChangeListener}\cite{propertychangelistener13}
listeners to be informed when the command finishes.
%
\end{asparadesc}

\begin{lstlisting}[style=Groovybash, label={lst:example_background1}, 
title={Starts the command in the background and waits for the command to finish 
at some later time.}]
def cmd = create "echo" theCommand
def task = cmd.background "Test"
task.get()
\end{lstlisting}

\begin{lstlisting}[style=Groovybash, label={lst:example_background2}, 
title={Inform when the command finishes in the background.}]
def mylistener = { evt -> echo "Done." } as PropertyChangeListener
def cmd = create "echo" theCommand
def task = cmd.background listener: mylistener, "Test"
\end{lstlisting}

\begin{lstlisting}[style=Groovybash, label={lst:example_background2}, 
title={Inform the listeners when the command finishes in the background.}]
def mylistenerA = { evt -> echo "Done A." } as PropertyChangeListener
def mylistenerB = { evt -> echo "Done B." } as PropertyChangeListener
cat.background listeners: [mylistenerA, mylistenerB], file
\end{lstlisting}

\pagebreak
\TheOperator{args()}%
\Operator{args(object...)}

The \Operator{args()} operator sets the arguments for the command, like 
the \Operator{call()} operator, but will not execute the command. The return
value is the command. The command can be executed at a later time by invoking
the \Operator{call()} operator.

\begin{lstlisting}[style=Groovybash, label={lst:example_args1}, title={Store 
the command in a variable with no arguments given.}]
def mycommand = command.args()
mycommand()
\end{lstlisting}

\begin{lstlisting}[style=Groovybash, label={lst:example_args2}, title={Store 
the command in a variable with arguments given either direct or by key=value 
pairs.}]
def mycat = cat.args file
def mycat = cat.args "-n", file
def mycat = cat.args "-n": file
mycat()
\end{lstlisting}

